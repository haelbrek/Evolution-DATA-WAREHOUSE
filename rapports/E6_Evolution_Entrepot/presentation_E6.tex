%%%%%%%%%%%%%%%%%%%%%%%%%%%%%%%%%%%%%%%%%%%%%%%%%%%%%%%%%%%%%%
% PRESENTATION E6 - EVOLUTION D'UN ENTREPOT DE DONNEES
% Soutenance Projet Data Engineering
% Auteur: Hamza Elbrek | Fevrier 2026
%%%%%%%%%%%%%%%%%%%%%%%%%%%%%%%%%%%%%%%%%%%%%%%%%%%%%%%%%%%%%%

\documentclass[aspectratio=169]{beamer}

\usepackage{fontspec}
\usepackage[french]{babel}
\usepackage{graphicx}
\usepackage{booktabs}
\usepackage{array}
\usepackage{tabularx}
\usepackage{tikz}
\usetikzlibrary{shapes.geometric, arrows, positioning, fit}
\usepackage{listings}
\usepackage{xcolor}
\usepackage{multicol}

% ===================== THEME =====================
\usetheme{Madrid}
\usecolortheme{default}

\definecolor{azure}{RGB}{0, 112, 192}
\definecolor{darkblue}{RGB}{0, 51, 102}
\definecolor{successgreen}{RGB}{39, 174, 96}
\definecolor{warnorange}{RGB}{230, 126, 34}
\definecolor{lightgray}{RGB}{245, 245, 245}
\definecolor{codebg}{RGB}{240, 240, 240}

\setbeamercolor{palette primary}{bg=darkblue, fg=white}
\setbeamercolor{palette secondary}{bg=azure, fg=white}
\setbeamercolor{palette tertiary}{bg=azure!80!black, fg=white}
\setbeamercolor{palette quaternary}{bg=darkblue, fg=white}
\setbeamercolor{structure}{fg=darkblue}
\setbeamercolor{title}{fg=white, bg=darkblue}
\setbeamercolor{frametitle}{fg=white, bg=darkblue}
\setbeamercolor{block title}{fg=white, bg=azure}
\setbeamercolor{block body}{fg=black, bg=azure!10}
\setbeamercolor{block title alerted}{fg=white, bg=warnorange}
\setbeamercolor{block body alerted}{fg=black, bg=warnorange!10}
\setbeamercolor{block title example}{fg=white, bg=successgreen!70!black}
\setbeamercolor{block body example}{fg=black, bg=successgreen!10}

\setbeamerfont{title}{size=\Large, series=\bfseries}
\setbeamerfont{frametitle}{size=\normalsize, series=\bfseries}

% Supprimer la barre de navigation
\setbeamertemplate{navigation symbols}{}

% Numerotation des slides
\setbeamertemplate{footline}{%
  \leavevmode%
  \hbox{%
    \begin{beamercolorbox}[wd=.33\paperwidth,ht=2.25ex,dp=1ex,center]{palette primary}%
      \usebeamerfont{author in head/foot}\insertshortauthor
    \end{beamercolorbox}%
    \begin{beamercolorbox}[wd=.34\paperwidth,ht=2.25ex,dp=1ex,center]{palette secondary}%
      \usebeamerfont{title in head/foot}\insertshorttitle
    \end{beamercolorbox}%
    \begin{beamercolorbox}[wd=.33\paperwidth,ht=2.25ex,dp=1ex,right]{palette primary}%
      \usebeamerfont{date in head/foot}
      \insertframenumber{} / \inserttotalframenumber\hspace*{2ex}
    \end{beamercolorbox}}%
}

% ===================== LISTINGS =====================
\lstset{
  backgroundcolor=\color{codebg},
  basicstyle=\ttfamily\tiny,
  breaklines=true,
  keywordstyle=\color{azure}\bfseries,
  commentstyle=\color{successgreen!70!black}\itshape,
  frame=single, framerule=0pt,
  xleftmargin=0.3cm, xrightmargin=0.3cm,
}

% ===================== INFOS =====================
\title[Projet E6 -- Data Warehouse]{%
  \textbf{Evolution d'un Entrepot de Donnees}\\
  \normalsize Projet Data Engineering E6 -- Region Hauts-de-France
}
\author[H. Elbrek]{Hamza Elbrek}
\institute[BTS SIO]{RNCP 37638: Expert en données massives}
\date{Mars 2026}

% ===================== LOGO (optionnel) =====================
% \logo{\includegraphics[height=0.8cm]{logo.png}}

%%%%%%%%%%%%%%%%%%%%%%%%%%%%%%%%%%%%%%%%%%%%%%%%%%%%%%%%%%%%%%
\begin{document}
%%%%%%%%%%%%%%%%%%%%%%%%%%%%%%%%%%%%%%%%%%%%%%%%%%%%%%%%%%%%%%

%-------------------------------------------------------
% SLIDE 1 : TITRE
%-------------------------------------------------------
\begin{frame}
\titlepage
\vspace{0.3cm}
\centering
\small
\textcolor{azure}{\rule{0.8\textwidth}{0.4pt}}\\[0.3cm]
\begin{columns}[c]
  \column{0.5\textwidth}
  \centering
  \textbf{Competences visees}\\
  \small C16 -- Exploiter des donnees\\
  \small C17 -- Gerer le patrimoine de donnees
  \column{0.5\textwidth}
  \centering
\end{columns}
\end{frame}

%-------------------------------------------------------
% SLIDE 2 : PLAN
%-------------------------------------------------------
\begin{frame}{Plan de la soutenance}
\tableofcontents
\end{frame}

\AtBeginSection[]{
  \begin{frame}{Sommaire}
    \tableofcontents[currentsection]
  \end{frame}
}

%=======================================================
\section{Contexte et objectifs}
%=======================================================

%-------------------------------------------------------
% SLIDE 3 : DU PROJET E5 AU PROJET E6
%-------------------------------------------------------
\begin{frame}{Du projet E5 au projet E6}

\begin{columns}[T]
  \column{0.48\textwidth}
  \begin{block}{Projet E5 -- Acquis}
    \begin{itemize}
      \item Entrepot Azure SQL deploye via Terraform
      \item Schema en etoile : 6 dimensions, 5 faits sur 7
      \item Pipelines ETL Python
      \item 3 roles RBAC initiaux
    \end{itemize}
  \end{block}

  \column{0.48\textwidth}
  \begin{alertblock}{Limites identifiees}
    \begin{itemize}
      \item Pas de journalisation structuree
      \item Pas de strategie de sauvegarde
      \item 2 tables de faits manquantes
      \item Aucune restriction geographique des acces
    \end{itemize}
  \end{alertblock}
\end{columns}

\vspace{0.4cm}
\begin{exampleblock}{Objectif E6}
  \centering
  Faire evoluer l'entrepot vers un \textbf{systeme effectif en conditions reelles} :\\
  maintenance, securite, historisation, nouvelles sources.
\end{exampleblock}

\end{frame}

%-------------------------------------------------------
% SLIDE 4 : INFRASTRUCTURE AZURE
%-------------------------------------------------------
\begin{frame}{Infrastructure Azure -- Vue d'ensemble}

\begin{columns}[T]
  \column{0.52\textwidth}
  \begin{block}{Ressources provisionnees (Terraform)}
    \begin{itemize}
      \item \textbf{Azure SQL Database} -- Data Warehouse
      \item \textbf{ADLS Gen2} -- Data Lake (raw / curated)
      \item \textbf{Azure Databricks} -- Traitement Spark
      \item \textbf{Azure Data Factory} -- Orchestration
      \item \textbf{Key Vault} -- Gestion des secrets
      \item \textbf{Log Analytics Workspace} -- Auditing SQL
    \end{itemize}
  \end{block}

  \column{0.44\textwidth}
  \begin{center}
  \includegraphics[width=\textwidth]{captures/azure_resource_group.png}
  \end{center}
\end{columns}

\end{frame}

%=======================================================
\section{Architecture de l'entrepot}
%=======================================================

%-------------------------------------------------------
% SLIDE 5 : SCHEMA EN ETOILE
%-------------------------------------------------------
\begin{frame}{Architecture -- Schema en etoile E6}

\begin{columns}[T]
  \column{0.52\textwidth}
  \begin{block}{5 schemas fonctionnels}
    \begin{itemize}
      \item \texttt{stg} -- Zone de staging (donnees brutes CSV)
      \item \texttt{dwh} -- Dimensions + Faits (cœur du DWH)
      \item \texttt{dm} -- Datamarts analytiques (vues materialisees)
      \item \texttt{analytics} -- Tableaux de bord + monitoring
      \item \textcolor{warnorange}{\texttt{security}} -- RBAC, RLS, agences, employes \textbf{[E6]}
    \end{itemize}
  \end{block}

  \column{0.44\textwidth}
  \begin{exampleblock}{Nouveautes E6}
    \small
    \begin{itemize}
      \item \textcolor{successgreen}{\texttt{fait\_emploi}} -- Chomage / emploi
      \item \textcolor{successgreen}{\texttt{fait\_menages}} -- Composition menages
      \item \textcolor{successgreen}{\texttt{log\_etl}} -- Journalisation ETL
      \item \textcolor{successgreen}{\texttt{log\_erreurs}} -- Erreurs pipeline
      \item Schema \textcolor{warnorange}{\texttt{security}} complet
    \end{itemize}
  \end{exampleblock}
\end{columns}

\vspace{0.2cm}
\centering
\includegraphics[width=0.6\textwidth]{captures/azure_datawarehouse_schema.png}

\end{frame}

%=======================================================
\section{Maintenance et journalisation}
%=======================================================

%-------------------------------------------------------
% SLIDE 6 : GESTION DES INCIDENTS
%-------------------------------------------------------
\begin{frame}{Gestion des incidents -- Processus ITIL}

\begin{columns}[T]
  \column{0.48\textwidth}
  \textbf{Workflow en 8 etapes :}
  \begin{enumerate}
    \small
    \item Reception mail $\rightarrow$ creation ticket
    \item Prise en charge (correctif ou escalade)
    \item Priorisation manager si necessaire
    \item Accuse de reception expediteur
    \item Diagnostic via vues de monitoring
    \item Resolution et tests
    \item Notification de cloture
    \item Archivage backlog
  \end{enumerate}

  \column{0.48\textwidth}
  \begin{block}{Matrice de priorisation}
    \small
    \begin{tabular}{|l|l|}
      \hline
      \textbf{P1 Critique} & Pipeline ETL en erreur \\
      \hline
      \textbf{P2 Haute} & Donnees incorrectes \\
      \hline
      \textbf{P3 Normale} & Demande evolution \\
      \hline
      \textbf{P4 Basse} & Documentation \\
      \hline
    \end{tabular}
  \end{block}
  \vspace{0.2cm}
  \begin{alertblock}{Regle cle}
    \small Aucune demande ne reste sans ticket ouvert.
  \end{alertblock}
\end{columns}

\end{frame}

%-------------------------------------------------------
% SLIDE 7 : JOURNALISATION
%-------------------------------------------------------
\begin{frame}{Journalisation -- Double niveau de tracabilite}

\begin{columns}[T]
  \column{0.44\textwidth}
  \begin{block}{Niveau 1 -- Logs Python}
    \small
    Module \texttt{logging} standard :\\
    \texttt{INFO}, \texttt{WARNING}, \texttt{ERROR}\\
    $\rightarrow$ Console + fichier local
  \end{block}
  \vspace{0.2cm}
  \begin{block}{Niveau 2 -- Table SQL}
    \small
    \texttt{dwh.log\_etl} enregistre :\\
    etape, table, statut, duree, nb lignes
  \end{block}
  \vspace{0.2cm}
  \begin{exampleblock}{Notifications email}
    \small
    \texttt{etl\_notifier.py} envoie un rapport HTML\\
    a chaque fin de pipeline (succes ou erreur)
  \end{exampleblock}

  \column{0.52\textwidth}
  \includegraphics[width=\textwidth]{captures/azure_log_etl.png}
  \vspace{0.15cm}
  \includegraphics[width=\textwidth]{captures/Notifications email automatiques du pipeline ETL.png}
\end{columns}

\end{frame}

%=======================================================
\section{Strategie de sauvegarde}
%=======================================================

%-------------------------------------------------------
% SLIDE 8 : BACKUP
%-------------------------------------------------------
\begin{frame}{Strategie de sauvegarde -- 2 niveaux}

\begin{columns}[T]
  \column{0.48\textwidth}
  \begin{block}{Niveau 1 -- PITR Azure (automatique)}
    \small
    \begin{itemize}
      \item Sauvegarde complete + differentielle toutes les 12h
      \item Retention court terme : \textbf{14 jours}
      \item Retention long terme : 4 sem / 12 mois / 3 ans
      \item Restauration a la minute pres (RTO $\approx$ 1h)
    \end{itemize}
  \end{block}
  \vspace{0.2cm}
  \begin{block}{Niveau 2 -- Export BACPAC (pipeline ETL)}
    \small
    \begin{itemize}
      \item Etape 5 du pipeline ETL
      \item Export \texttt{.bacpac} horodate vers ADLS Gen2
      \item Container \texttt{raw/backups/}
      \item Nettoyage auto des fichiers $>$ 30 jours
    \end{itemize}
  \end{block}

  \column{0.48\textwidth}
  \includegraphics[width=\textwidth]{captures/azure_backup_config.png}
  \vspace{0.2cm}
  \includegraphics[width=\textwidth]{captures/azure_datalake_bacpac.png}
\end{columns}

\end{frame}

%=======================================================
\section{Securite des acces (RBAC + RLS)}
%=======================================================

%-------------------------------------------------------
% SLIDE 9 : RBAC
%-------------------------------------------------------
\begin{frame}{Refonte des roles -- RBAC E6}

\begin{columns}[T]
  \column{0.52\textwidth}
  \begin{block}{4 roles metier (E6)}
    \small
    \begin{tabular}{|l|l|}
      \hline
      \texttt{role\_admin} & Controle total \\
      \hline
      \texttt{role\_etl\_process} & R/W stg + dwh \\
      \hline
      \texttt{role\_analyst} & Lecture dwh/dm/analytics \\
      \hline
      \texttt{role\_consultant} & \textbf{Lecture filtree par RLS} \\
      \hline
    \end{tabular}
  \end{block}
  \vspace{0.2cm}
  \begin{alertblock}{Changement cle}
    \small Suppression de \texttt{role\_dwh\_admin} et introduction\\
    de \texttt{role\_consultant} avec filtrage geographique.
  \end{alertblock}

  \column{0.44\textwidth}
  \includegraphics[width=\textwidth]{captures/azure_rbac_roles.png}
  \vspace{0.2cm}
  \begin{exampleblock}{Referentiel employes}
    \small
    \textbf{519 employes} fictifs generes\\
    (contrainte RGPD -- pas d'import\\
    de donnees bancaires reelles sur Azure)
  \end{exampleblock}
\end{columns}

\end{frame}

%-------------------------------------------------------
% SLIDE 10 : RLS
%-------------------------------------------------------
\begin{frame}{Row-Level Security -- Filtrage geographique}

\begin{columns}[T]
  \column{0.52\textwidth}
  \begin{block}{Architecture RLS}
    \small
    \begin{itemize}
      \item Predicat applique sur \texttt{dwh.dim\_geographie}
      \item Propagation automatique a toutes les vues \texttt{dm.*} et \texttt{analytics.*}
      \item Fonction \texttt{security.fn\_rls\_geographie}
    \end{itemize}
  \end{block}
  \vspace{0.2cm}
  \begin{block}{Hierarchie -- 4 niveaux}
    \small
    \begin{tabular}{|l|r|}
      \hline
      Directeur Regional & 1 \\
      \hline
      Directeurs Departement & 5 \\
      \hline
      Directeurs Agence & 101 \\
      \hline
      Collaborateurs & 412 \\
      \hline
    \end{tabular}
  \end{block}

  \column{0.44\textwidth}
  \includegraphics[width=\textwidth]{captures/RLS.png}
  \vspace{0.2cm}
  \begin{exampleblock}{101 agences}
    \small Communes Hauts-de-France\\
    $>$ 10 000 habitants\\
    (02, 59, 60, 62, 80)
  \end{exampleblock}
\end{columns}

\end{frame}

%=======================================================
\section{Variations de dimensions (SCD)}
%=======================================================

%-------------------------------------------------------
% SLIDE 11 : SCD
%-------------------------------------------------------
\begin{frame}{Slowly Changing Dimensions -- 3 types implementes}

\begin{columns}[T]
  \column{0.32\textwidth}
  \begin{block}{SCD Type 1}
    \small \textbf{Ecrasement}\\
    \vspace{0.1cm}
    Correction de libelle NAF.\\
    L'historique est perdu.\\
    \vspace{0.1cm}
    $\rightarrow$ \texttt{dim\_activite}
  \end{block}

  \column{0.32\textwidth}
  \begin{exampleblock}{SCD Type 2}
    \small \textbf{Historisation}\\
    \vspace{0.1cm}
    Fusion de communes.\\
    Colonnes \texttt{est\_actif},\\
    \texttt{version}, \texttt{date\_debut}.\\
    \vspace{0.1cm}
    $\rightarrow$ \texttt{dim\_geographie}
  \end{exampleblock}

  \column{0.32\textwidth}
  \begin{alertblock}{SCD Type 3}
    \small \textbf{Colonne precedente}\\
    \vspace{0.1cm}
    Renommage categorie PCS.\\
    1 seul niveau d'historique\\
    conserve.\\
    \vspace{0.1cm}
    $\rightarrow$ \texttt{dim\_demographie}
  \end{alertblock}
\end{columns}

\vspace{0.3cm}
\centering
\includegraphics[width=0.65\textwidth]{captures/azure_scd_dim_geographie.png}

\end{frame}

%=======================================================
\section{Bilan et competences}
%=======================================================

%-------------------------------------------------------
% SLIDE 12 : BILAN TECHNIQUE
%-------------------------------------------------------
\begin{frame}{Bilan technique -- Ce qui a ete realise}

\begin{columns}[T]
  \column{0.48\textwidth}
  \textbf{Maintenance \& monitoring}
  \begin{itemize}
    \small
    \item[$\checkmark$] Processus incidents (8 etapes)
    \item[$\checkmark$] Table \texttt{dwh.log\_etl} + \texttt{log\_erreurs}
    \item[$\checkmark$] Notifications email automatiques
    \item[$\checkmark$] Vues analytics de supervision
  \end{itemize}
  \vspace{0.2cm}
  \textbf{Sauvegarde}
  \begin{itemize}
    \small
    \item[$\checkmark$] PITR Azure 14 jours + retention LT
    \item[$\checkmark$] Export BACPAC vers ADLS Gen2
    \item[$\checkmark$] Nettoyage automatique
  \end{itemize}

  \column{0.48\textwidth}
  \textbf{Securite}
  \begin{itemize}
    \small
    \item[$\checkmark$] 4 roles RBAC metier
    \item[$\checkmark$] RLS sur \texttt{dwh.dim\_geographie}
    \item[$\checkmark$] 519 employes / 101 agences charges
    \item[$\checkmark$] Schema \texttt{security} dedie
  \end{itemize}
  \vspace{0.2cm}
  \textbf{Evolution du modele}
  \begin{itemize}
    \small
    \item[$\checkmark$] 2 nouvelles tables de faits
    \item[$\checkmark$] SCD Type 1, 2 et 3 implementes
    \item[$\checkmark$] 7 tables de faits completees
  \end{itemize}
\end{columns}

\end{frame}

%-------------------------------------------------------
% SLIDE 13 : COMPETENCES C16/C17
%-------------------------------------------------------
\begin{frame}{Competences visees -- C16 et C17}

\begin{columns}[T]
  \column{0.48\textwidth}
  \begin{block}{C16 -- Exploiter des donnees}
    \small
    \begin{itemize}
      \item Integration de nouvelles sources CSV
      \item Requetage SQL analytique (datamarts)
      \item Vues de monitoring et supervision
      \item Row-Level Security sur les consultations
    \end{itemize}
  \end{block}

  \column{0.48\textwidth}
  \begin{block}{C17 -- Gerer le patrimoine de donnees}
    \small
    \begin{itemize}
      \item Politique de sauvegarde et restauration
      \item Journalisation et tracabilite des ETL
      \item Gestion des roles et des acces (RBAC)
      \item Historisation SCD Type 1/2/3
      \item Procedures de scalabilite documentees
    \end{itemize}
  \end{block}
\end{columns}

\vspace{0.4cm}
\begin{exampleblock}{Valeur ajoutee}
  \centering
  \small
  Un entrepot de donnees passe de l'\textbf{etat initial E5} a un \textbf{systeme operationnel complet} :\\
  monitore, securise, sauvegarde et pret a evoluer.
\end{exampleblock}

\end{frame}

%-------------------------------------------------------
% SLIDE 14 : CONCLUSION
%-------------------------------------------------------
\begin{frame}{Conclusion}

\begin{center}
\Large\textbf{Merci pour votre attention}

\vspace{0.5cm}
\textcolor{azure}{\rule{0.6\textwidth}{0.4pt}}
\vspace{0.5cm}

\normalsize
\begin{tabular}{ll}
  \textbf{Auteur}    & Hamza Elbrek \\
  \textbf{Projet}    & E6 -- Evolution d'un Entrepot de Donnees \\
  \textbf{Region}    & Hauts-de-France \\
  \textbf{Stack}     & Azure SQL $\cdot$ Python $\cdot$ Terraform $\cdot$ ADLS Gen2 \\
  \textbf{Date}      & Fevrier 2026 \\
\end{tabular}

\vspace{0.5cm}
\textcolor{azure}{\rule{0.6\textwidth}{0.4pt}}

\vspace{0.4cm}
\Large \textbf{Questions ?}
\end{center}

\end{frame}

%%%%%%%%%%%%%%%%%%%%%%%%%%%%%%%%%%%%%%%%%%%%%%%%%%%%%%%%%%%%%%
\end{document}
